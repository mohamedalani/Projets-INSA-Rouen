\hypertarget{personnage_8cpp}{\section{/nfs/usersgm/\-G\-M26/neljibbawe/\-Bureau/\-Projet\-C\-P\-P\-A\-Rendre/\-C\-P\-P R\-P\-G/\-R\-P\-G/personnage.cpp File Reference}
\label{personnage_8cpp}\index{/nfs/usersgm/\-G\-M26/neljibbawe/\-Bureau/\-Projet\-C\-P\-P\-A\-Rendre/\-C\-P\-P R\-P\-G/\-R\-P\-G/personnage.\-cpp@{/nfs/usersgm/\-G\-M26/neljibbawe/\-Bureau/\-Projet\-C\-P\-P\-A\-Rendre/\-C\-P\-P R\-P\-G/\-R\-P\-G/personnage.\-cpp}}
}
{\ttfamily \#include \char`\"{}personnage.\-hpp\char`\"{}}\\*
{\ttfamily \#include $<$string$>$}\\*
{\ttfamily \#include $<$vector$>$}\\*
{\ttfamily \#include $<$iostream$>$}\\*
